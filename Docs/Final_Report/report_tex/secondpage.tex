% !TEX root = /Users/kartikgohil/Documents/Imperial/Year4/Project/Docs/Final_Report/report_tex/main.tex
\thispagestyle{empty}

\begin{tabular}{|l}

\end{tabular}


%%%%%%%%%%%%%%%%%%%%%%%%%%%%%%%%%%%%%%%%%%%%%
\vspace{150 pt}
\begin{center}
\huge Programmable And Wearable Sound\\
\LARGE  (PAWS)\\
\bigskip
\LARGE by Kartiksinh Gohil

\end{center}

%%%%%%%%%%%%%%%%%%%%%%%%%%%%%%%%%%%%%%%%%%%%%
\vspace{100 pt}
\begin{center}
\textbf{Acknowledgements}\\
\textit{Special thanks to Imperial College Advanced Hackspace for their equipment,\\Vic Boddy for his assistance, and my supervisors for their advice.}\\

\end{center}
%%%%%%%%%%%%%%%%%%%%%%%%%%%%%%%%%%%%%%%%%%%%%

\vfill
\normalsize
%%%%%%%%%%%%%%%%%%%%%%%%%%%%%%%%%%%%%%%%%%%%%
\section*{Abstract} \label{Project Specification}

The aim of the project was the development of a novel wearable musical instrument whose functionality is flexible and fully programmable through a simple interface.

This report describes the design of a concept musical instrument in which the user wears sensors that can be programmed, through a software application, to generate sounds selected by that user. The concept is very general, and is illustrated by a `demonstration-of-concept' prototype that has been developed and evaluated.  The prototype focuses on one representative function whereby the user, while wearing microphones on their fingers, can play a rhythm against an arbitrary surface: simultaneously, and in response, the software will play back previously selected sample sound files. For example, a user can play a full drum kit simply with their fingers, against a musical background of their choosing or as a solo performance.

Various mechanisms were developed to capture sound using a microphone, and to transmit this audio signal to a desktop application via an Arduino Uno. Python and C++ were both tested for creating the software interface, and the latter, using an audio processing library called JUCE, was finally selected for the later prototypes. A program was developed to read audio signals from a single Arduino Uno connected to three wearable circuit boards, each with their own microphone circuit, and was designed to control the sample sound file triggered by each. The user experience was a main consideration in the design of the instrument and the program included features such as a real-time waveform trace of the incoming audio signals to aid this. Further designs were subsequently looked into for improved technical implementation and greater functional creativity for the user.

%%%%%%%%%%%%%%%%%%%%%%%%%%%%%%%%%%%%%%%%%%%%%
\normalsize
\newpage